% !TEX root = HW1.tex
\documentclass[11pt,epic]{article}
\textwidth=6.5in
\textheight=8.5in
\usepackage{url}
\usepackage{hyperref}

\usepackage{mathtools,amssymb,bm}
\usepackage{booktabs}
\usepackage{psfrag}
\usepackage{amsthm}
\usepackage{epstopdf}
\usepackage{graphicx}
\usepackage{enumitem}
\usepackage{latexsym}
\usepackage{setspace}
\usepackage{fullpage}
\usepackage{algorithmic}
\usepackage{algorithm}
\usepackage{pdfpages}
\def\argmin{\operatornamewithlimits{arg\, min}}
\usepackage{color}
\usepackage{framed}

\usepackage{listings}
\usepackage{pdfpages}

\definecolor{codegreen}{rgb}{0,0.6,0}
\definecolor{codegray}{rgb}{0.5,0.5,0.5}
\definecolor{codepurple}{rgb}{0.58,0,0.82}
\definecolor{backcolour}{rgb}{0.95,0.95,0.92}

% \lstdefinestyle{mystyle}{
%     backgroundcolor=\color{backcolour},   
%     commentstyle=\color{codegreen},
%     keywordstyle=\color{magenta},
%     numberstyle=\tiny\color{codegray},
%     stringstyle=\color{codepurple},
%     basicstyle=\ttfamily\footnotesize,
%     breakatwhitespace=false,         
%     breaklines=true,                 
%     captionpos=b,                    
%     keepspaces=true,                 
%     numbers=left,                    
%     numbersep=5pt,                  
%     showspaces=false,                
%     showstringspaces=false,
%     showtabs=false,                  
%     tabsize=2
% }

\lstset{
    basicstyle=\footnotesize\ttfamily, % Use a smaller font size
    numbers=left,                     % Line numbers on the left
    numberstyle=\tiny,                % Smaller line number size
    frame=single,                     % Frame around the code
    breaklines=true                   % Enable line breaking
}

% \definecolor{gray}{rgb}{0.3,0.3,0.3}
\definecolor{gray}{rgb}{0.7,0.01,0.2}
\definecolor{darkgreen}{rgb}{0,0.55,0}
\definecolor{purple}{rgb}{0.5,0,1}
\raggedbottom
\newenvironment{solution}
  {\par\medskip
  \color{gray}%
   \begin{framed}
   \textbf{Solution: \newline}\ignorespaces}
 {\end{framed}
  \medskip}


  
\begin{document}

\title{CSE6740 CDA Homework 1}
\author{
Name:  \\
GTID: 
}
\date{Deadline: Sep 14th 11:59 pm ET}
\maketitle

\section{Principal Component Analysis (PCA) [15 pts]}

\subsection{Reconstruction error and variance maximization [5 pts] }

Assume the centered data matrix $x\in\mathbb{R}^{n\times d}$ has rows $\mathbf{x}_1^\top,\ldots,\mathbf{x}_n^\top$. Let $q\le d$ and choose $q$ unit-length, mutually orthogonal directions $\mathbf{w}_1,\ldots,\mathbf{w}_q\in\mathbb{R}^d$. Define $w=[\mathbf{w}_1\ \cdots\ \mathbf{w}_q]\in\mathbb{R}^{d\times q}$.

\begin{enumerate}
\item[(a)] Write $w$ as the matrix formed by stacking the $\mathbf{w}_i$. Prove that $w^\top w = I_q$.

\item[(b)] \textbf{Scores:} Find the $n\times q$ matrix of $q$-dimensional scores in terms of $x$ and $w$. \textit{Hint: your answer should reduce to $\mathbf{x}_i^\top\mathbf{w}_1$ when $q=1$.}

\item[(c)] \textbf{Reconstructions:} Find the $n\times d$ matrix of $d$-dimensional approximations (reconstructions) based on these scores, in terms of $x$ and $w$. \textit{Hint: your answer should reduce to $(\mathbf{x}_i^\top\mathbf{w}_1)\mathbf{w}_1^\top$ when $q=1$.}

\item[(d)] Show that the mean-squared reconstruction error obtained using $\mathbf{w}_1,\ldots,\mathbf{w}_q$ is the sum of two terms: one that depends only on $x$ (and not on $w$), and another that depends only on the scores along those directions (and not otherwise on what those directions are). 

\item[(e)] Explain in what sense minimizing projection residuals is equivalent to maximizing the sum of the variances of the scores along the different directions.
\end{enumerate}

\begin{solution}

\end{solution}


\subsection{First Principal Component via Projection-Error Minimization [10pts]} 

Let $\{\mathbf{x}_1,\ldots,\mathbf{x}_n\}\subset\mathbb{R}^d$ be centered and standardized so each coordinate has unit variance. For a unit vector $\mathbf{v}$, define $f_{\mathbf{v}}(\mathbf{x})$ to be the projection of $\mathbf{x}$ onto the line $\mathcal{V}=\{\alpha\mathbf{v}:\alpha\in\mathbb{R}\}$, i.e.,
\[
f_{\mathbf{v}}(\mathbf{x})=\arg\min_{\mathbf{u}\in\mathcal{V}}\|\mathbf{x}-\mathbf{u}\|^2
\]
Show that the unit vector minimizing the mean squared projection error,
\[
\arg\min_{\|\mathbf{v}\|_2=1}\ \sum_{i=1}^n \bigl\|\mathbf{x}_i - f_{\mathbf{v}}(\mathbf{x}_i)\bigr\|_2^2,
\]
is the first principal component of the data.


\begin{solution}

\end{solution}

\section{Density Estimation}
\subsection{Call Center Counts and Likelihood-based Estimation [5pts]}  

A call center records the number of customer calls per hour. Over one week you collect data for 70 hours. Typical hourly counts are: 6, 9, 7, 11, $\ldots$  

\begin{enumerate}
    \item Suggest an appropriate probability distribution for this data and justify your choice.  
    \item Find the parameter estimate that makes the observed data most probable under your chosen distribution.  
\end{enumerate}  

\begin{solution}

\end{solution}  

---

\subsection{Pareto Distribution Parameter Estimation [5pts]}  

The Pareto distribution has been used in economics to model income and wealth distributions with slowly decaying tails. Its density is given by  
\[
f(x \mid x_0, \theta) = \theta x_0^{\theta} x^{-(\theta+1)}, \quad x \geq x_0,\ \theta > 0,
\]  
where $x_0$ is known. Based on $n$ i.i.d. samples $x_1,\ldots,x_n$, find the estimator of $\theta$ that maximizes the likelihood function.  

\begin{solution}

\end{solution}  

---

\subsection{Parametric vs Nonparametric Density Choice [5pts]}  

Consider the following two figures showing empirical density plots.  

\begin{center}
\includegraphics[width=0.7\linewidth]{figureA.jpeg} \\
\textbf{Figure A}  

\vspace{0.5cm}
\includegraphics[width=0.7\linewidth]{figureB.jpeg} \\ 
\textbf{Figure B}  
\end{center}
\begin{enumerate}
    \item For Figure A, decide whether a \textbf{parametric} or \textbf{nonparametric} density estimation method is more appropriate. Justify your answer and suggest a suitable method.  
    \item Repeat the same question for Figure B.  
\end{enumerate}  

\begin{solution}

\end{solution}  


\section{Expectation-Maximization (EM) Algorithm [20 pts]}
\subsection{Relation to K-means [8pts]}
Consider a Gaussian mixture model in $\mathbb{R}^d$ with fixed covariance matrices $\epsilon \mathbf{I}$ for all the mixture components, where $\mathbf{I}$ is the identity matrix and $\epsilon > 0$ is a given constant. The class-conditional density is  
\begin{equation*}
    p(\mathbf{x}|\boldsymbol{\mu}_k, \boldsymbol{\Sigma}_k)=
    \frac{1}{(2\pi\epsilon)^{d/2}}\exp\left({-\frac{1}{2\epsilon}\|\mathbf{x}-\boldsymbol{\mu}_k\|^2}\right).
\end{equation*}
We now consider the EM algorithm for a mixture of K Gaussians of this form while treating $\epsilon$ as a fixed constant instead of a parameter to estimate. Let $\{\pi_k\}_{k=1}^K$ be the mixing proportions and $\{\mathbf{x}^i\}_{i=1}^n$ the data points. The latent variable $z_i$ indicates the Gaussian component to which the data point $\mathbf{x}^i$ belongs.
\begin{enumerate}[label=(\alph*)]
    \item Write down the posterior probability $\tau_k^i=p(z_i=k|\mathbf{x}^i)$ in this setting.
    \begin{solution}
        
    \end{solution}


    \item Show that in the limit $\epsilon \rightarrow 0$, maximizing the expected complete-data log likelihood \begin{equation*}
        \mathbb{E}_{z_1\sim p(z_1|\mathbf{x}^1), ...,z_n\sim p(z_n|\mathbf{x}^n)}\left[\log{\prod_{i=1}^n p(\mathbf{x}^i, z_i|\boldsymbol{\pi}, \boldsymbol{\mu})}\right]
    \end{equation*} is equivalent to minimizing the K-means objective $\frac{1}{n}\sum_{i=1}^n\left\|\mathbf{x}^i-\boldsymbol{\mu}_{c(i)}\right\|^2$, where $c(i)=\argmin_k \left\|\mathbf{x}^i-\boldsymbol{\mu}_k\right\|^2$.
    \begin{solution}
        
    \end{solution}
    \end{enumerate}

\subsection{Mixture of Exponential Distributions [12pts]}
In this question, you will extend the EM algorithm to the mixture of exponential distributions. Suppose we have $N$ i.i.d samples $x_1, x_2, ..., x_N\in[0, \infty)$ drawn from a mixture of $K$ exponential components. Let mixture weights $\boldsymbol{\pi}=\left(\pi_1, \pi_2, ..., \pi_K\right)$ satisfy $\pi_k \geq 0$ and $\sum_{k=1}^K\pi_k=1$, and component rates $\boldsymbol{\lambda}=\left(\lambda_1, \lambda_2, ..., \lambda_K\right)$ satisfy $\lambda_k>0$. Given the component $k$, the likelihood of observing an instance $x$ is $P(x|k)=\lambda_ke^{-\lambda_kx}$.

\begin{enumerate}[label=(\alph*)]
    \item Write down the log-likelihood $L(\boldsymbol{\pi}, \boldsymbol{\lambda})$ for $N$ observations $x_1, x_2, ..., x_N$.
    \begin{solution}
        
    \end{solution}
    \item Given $\boldsymbol{\pi}$ and $\boldsymbol{\lambda}$, derive the lower bound of the log-likelihood $L(\boldsymbol{\pi}, \boldsymbol{\lambda})$ using Jensen's inequality, and write down the update rule for E-step.
    \begin{solution}
        
    \end{solution}
    \item Write down the M-step which maximizes your lower bound written above. To receive full credits, you should provide the answer step by step.
    \begin{solution}
        
    \end{solution}
\end{enumerate}


\section{Programming}
Please use \href{https://drive.google.com/drive/folders/1YECQBUdEhWzw2gfg_ZYUWyohJu2xH1Ws?usp=drive_link}{this link} to download all the required files. This homework contains only a ipynb, which you can make a copy and run on Google Colab
\subsection*{Deliverables}
For the programming part, please submit your \texttt{.ipynb} file to the programming autograder. Then, use \texttt{File} (top-left corner) → \texttt{Print} to generate and submit a PDF.  

Expected files
\begin{itemize}
    \item \texttt{HW1.pdf}
    \item \texttt{HW1.ipynb}
    \item \texttt{hw1.ipynb - Colab.pdf}
\end{itemize}


\bibliographystyle{elsarticle-num} 

\end{document}
